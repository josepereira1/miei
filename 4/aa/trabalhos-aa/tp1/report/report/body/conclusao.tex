\chapter{Arquitetura Tipo}

\hspace{5mm} Com uma análise detalhada, foram escolhidas as frameworks \textbf{Grails} e \textbf{Spring} para a proposta de arquitetura tipo. A framework \textbf{Grails} será utilizada para o frontend (client-side) da aplicação, por outro lado, a framework \textbf{Spring} será utilizada para o backend (server-side).

\hspace{5mm} Na verdade, se na framework \textbf{GWT} fossem acrescentados outros packages (tal como no Spring foi adicionado o \textbf{Spring Boot} para melhorar a framework), seria possível ter o client-side e o server-side com a \textbf{GWT} e através do \textbf{Hibernate} realizar o acesso à base de dados. No entanto, a framework Spring é bastante mais simples que a \textbf{GWT} com os packages extra. Desta forma, fica mais fácil integrar a mesma com o Hibernate, ficando um backend completo e simples.


\begin{minipage}{0.5\textwidth}
\begin{figure}[H]
\hfill
\includegraphics[scale=0.27]{images/grails.png}
\label{fig:cgrails}
\end{figure}
\end{minipage} \hfill
\begin{minipage}{0.45\textwidth}
FrontEnd
\end{minipage}

\begin{minipage}{0.5\textwidth}
\begin{figure}[H]
\hfill
\includegraphics[scale=0.11]{images/spring.png}
\label{fig:cspring}
\end{figure}
\end{minipage} \hfill
\begin{minipage}{0.45\textwidth}
BackEnd
\end{minipage}

\begin{minipage}{0.5\textwidth}
\begin{figure}[H]
\hfill
\includegraphics[scale=0.25]{images/hibernate.jpg}
\label{fig:chibernate}
\end{figure}
\end{minipage} \hfill
\begin{minipage}{0.45\textwidth}
DataBase
\end{minipage}


\chapter{Conclusão}
\label{sec:conclusao}

\hspace{5mm} Em suma, após a conclusão deste trabalho, percebemos a grande variedade de frameworks Java Web existentes. Desta forma, escolheu-se para análise, aquelas que cumprem os requisitos do enunciado, sendo também as que nos pareceram menos complexas de analisar, contendo melhor documentação, bem como um boa comunidade para ajuda.

\hspace{5mm} Algumas frameworks que analisamos, mas que não estão presentes no relatório, tais como \textbf{Blade}, \textbf{Struts}, entre outras, deve-se ao facto de serem complexas, ou de não conterem uma boa documentação, sendo difícil a sua análise. No entanto, com mais tempo, provavelmente o grupo irá, experimentá-las.

\hspace{5mm}Do mesmo modo, o grupo decidiu não análisar frameworks .NET, pois, apesar de já ter sido usado por exemplo ASP.NET em Laboratórios de Informática 4, achou-se mais interessante as frameworks de Java, visto a maior familarização com a linguagem.
